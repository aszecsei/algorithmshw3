\documentclass{article}

\usepackage[T1]{fontenc}

\usepackage{fancyhdr}
\usepackage{extramarks}
\usepackage{amsmath}
\usepackage{amsthm}
\usepackage{amsfonts}
\usepackage{tikz}
\usepackage{algorithm}
\usepackage{algpseudocode}
\usepackage{enumitem}

\usepackage[mono=false]{libertine}


\usetikzlibrary{automata,positioning}

%
% Basic Document Settings
%

\topmargin=-0.45in
\evensidemargin=0in
\oddsidemargin=0in
\textwidth=6.5in
\textheight=9.0in
\headsep=0.25in

\linespread{1.1}

\pagestyle{fancy}
\lhead{\hmwkAuthorName}
\chead{\hmwkClass: \hmwkTitle}
\rhead{\firstxmark}
\lfoot{\lastxmark}
\cfoot{\thepage}

\renewcommand\headrulewidth{0.4pt}
\renewcommand\footrulewidth{0.4pt}

\setlength\parindent{0pt}

%
% Create Problem Sections
%

\newcommand{\enterProblemHeader}[1]{
    \nobreak\extramarks{}{Problem \arabic{#1} continued on next page\ldots}\nobreak{}
    \nobreak\extramarks{Problem \arabic{#1} (continued)}{Problem \arabic{#1} continued on next page\ldots}\nobreak{}
}

\newcommand{\exitProblemHeader}[1]{
    \nobreak\extramarks{Problem \arabic{#1} (continued)}{Problem \arabic{#1} continued on next page\ldots}\nobreak{}
    \stepcounter{#1}
    \nobreak\extramarks{Problem \arabic{#1}}{}\nobreak{}
}

\setcounter{secnumdepth}{0}
\newcounter{partCounter}
\newcounter{homeworkProblemCounter}
\setcounter{homeworkProblemCounter}{1}
\nobreak\extramarks{Problem \arabic{homeworkProblemCounter}}{}\nobreak{}

%
% Homework Problem Environment
%
% This environment takes an optional argument. When given, it will adjust the
% problem counter. This is useful for when the problems given for your
% assignment aren't sequential. See the last 3 problems of this template for an
% example.
%
\newenvironment{homeworkProblem}[1][-1]{
    \ifnum#1>0
        \setcounter{homeworkProblemCounter}{#1}
    \fi
    \section{Problem \arabic{homeworkProblemCounter}}
    \setcounter{partCounter}{1}
    \enterProblemHeader{homeworkProblemCounter}
}{
    \exitProblemHeader{homeworkProblemCounter}
}

%
% Homework Details
%   - Title
%   - Due date
%   - Class
%   - Section/Time
%   - Instructor
%   - Author
%

\newcommand{\hmwkTitle}{Homework\ \#3}
\newcommand{\hmwkDueDate}{March 1, 2018}
\newcommand{\hmwkClass}{Design and Analysis of Algorithms}
\newcommand{\hmwkClassInstructor}{Professor Kasturi Varadarajan}
\newcommand{\hmwkAuthorName}{\textbf{Alic Szecsei}}

%
% Title Page
%

\title{
    \vspace{2in}
    \textmd{\textbf{\hmwkClass:\ \hmwkTitle}}\\
    \normalsize\vspace{0.1in}\small{Due\ in\ class\ on\ \hmwkDueDate}\\
    \vspace{0.1in}\large{\textit{\hmwkClassInstructor}}
    \vspace{3in}
}

\author{\hmwkAuthorName}
\date{}

\renewcommand{\part}[1]{\textbf{\large Part \Alph{partCounter}}\stepcounter{partCounter}\\}

%
% Various Helper Commands
%

% Useful for algorithms
\newcommand{\alg}[1]{\textsc{\bfseries \footnotesize #1}}

% For derivatives
\newcommand{\deriv}[1]{\frac{\mathrm{d}}{\mathrm{d}x} (#1)}

% For partial derivatives
\newcommand{\pderiv}[2]{\frac{\partial}{\partial #1} (#2)}

% Integral dx
\newcommand{\dx}{\mathrm{d}x}

% Alias for the Solution section header
\newcommand{\solution}{\textbf{\large Solution}}

% Probability commands: Expectation, Variance, Covariance, Bias
\newcommand{\E}{\mathrm{E}}
\newcommand{\Var}{\mathrm{Var}}
\newcommand{\Cov}{\mathrm{Cov}}
\newcommand{\Bias}{\mathrm{Bias}}

% Set from 1 to N
\newcommand{\XYZ}[1]{\left\{1,\ldots,{#1}\right\}}
\newcommand{\Break}{\textbf{break} }

\begin{document}

\maketitle

\pagebreak

\begin{homeworkProblem}
You and your eight-year-old nephew Elmo decide to play a simple card game. At the beginning of the game, the cards are dealt face up in a long row. Each card is worth a different number of points. After all the cards are dealt, you and Elmo take turns removing either the leftmost or rightmost card from the row, until all the cards are gone. At each turn, you can decide which of the two cards to take. The winner of the game is the player that has collected the most points when the game ends.\\

Having never taken an algorithms class, Elmo follows the obvious greedy strategy---when it's his turn, Elmo always takes the card with the higher point value. Your task is to find a strategy that will beat Elmo whenever possible. (It might seem mean to beat up on a little kid like this, but Elmo absolutely hates it when grown-ups let him win.

\begin{enumerate}[label=(\alph*)]
	\item Prove that you should not also use the greedy strategy. That is, show that there is a game that you can win, but only if you do \emph{not} follow the same greedy strategy as Elmo.
	\item Describe and analyze an algorithm to determine, given the initial sequence of cards, the maximum number of points that you can collect playing against Elmo.
\end{enumerate}

\solution\\

\part\\

Suppose there are a total of 8 cards, with point values ranging from 1 to 8. When dealt, they are arranged like so:

\begin{table}[h]
	\centering
	\begin{tabular}{ | *{8}{c |} c | }
		\hline
		3 & 8 & 7 & 6 & 5 & 4 & 2 & 1 \\
		\hline
\end{tabular}
	\caption{an example game}
	\label{tab:ExampleGame}
\end{table}

If you were to play using the greedy strategy, you would first take card 3, while Elmo would take 8, and so on. At the end of the game, you would have $3 + 7 + 5 + 2 = 17$ points, while Elmo would have $8 + 6 + 4 + 1 = 19$ points.\\

However, if you first took card 1, Elmo would take card 3, leaving you to take card 8, and so on, now playing greedily. At the end of the game, you would have $1 + 8 + 6 + 4 = 19$ points, while Elmo would have $3 + 7 + 5 + 2 = 17$ points.\\

Thus, you can only win this game by making your first move take the smaller of the two available cards, as opposed to Elmo's greedy strategy.\\

\pagebreak
\part\\

\begin{algorithm}
	\caption{Elmo's Card Game}\label{algo:elmosgame}
	\begin{algorithmic}[1]
		\Function{Elmo}{$cards[1..n]$}
			\State{$memorized \gets$ empty hash map}
			\State $memorized[$the empty list$] \gets 0$
			\For{$i \gets 1, n$}
				\For{$j \gets 1, n-i$}
					\State $subproblem \gets cards[j..j+i]$
					\If{$i = 1$}
						\State $memorized[subproblem] \gets subproblem[1]$
					\Else
						\State $L \gets subproblem[2..i]$ \Comment Try taking the first card
						\If{the first element of $L$ is greater than the last element of $L$}
							\State remove the first element of $L$
						\Else
							\State remove the last element of $L$
						\EndIf
						\State $l \gets subproblem[1] + memorized[L]$
						\State $R \gets subproblem[1..i-1]$ \Comment Try taking the last card
						\If{the first element of $R$ is greater than the last element of $R$}
							\State remove the first element of $R$
						\Else
							\State remove the last element of $R$
						\EndIf
						\State $r \gets subproblem[i] + memorized[R]$
						\State $memorized[subproblem] \gets$ the maximum of $l$ and $r$
					\EndIf
				\EndFor
			\EndFor
			\State \Return $memorized[cards]$
		\EndFunction
	\end{algorithmic}
\end{algorithm}

This algorithm works from the bottom up to determine all sub-problems. Essentially, at any stage of the game, we may have some $i$ cards, where $i < n$, and all $i$ cards were originally neighbors (since players may only remove cards to the furthest left and furthest right). Therefore, all possible sub-problems are simply combinations of $i$ neighboring cards. For each sub-problem, assuming it is not trivial (no cards remaining or one card remaining), we can perform two calculations: one assuming the player removes a card from the left, and one assuming she removes a card from the right. Whichever of these two routes produces the largest score is stored as the solution to the sub-problem. When all possible sub-problems have been calculated, the algorithm returns the solution to the $n$-sized sub-problem, which is the original problem.\\

The run-time here is straightforward to analyze; there are $n$ ``layers'' of sub-problems, since we are analyzing games of 1 remaining card, 2 remaining cards, and so on until we have reached the original problem of $n$ remaining cards. For each of these ``layers'', we find all possible collections of neighboring cards; so, in total, we have $\sum\limits_{i = 1}^{n} i$ sub-problems, which leaves us with a runtime of $O(n^2)$.

\end{homeworkProblem}

\pagebreak

\begin{homeworkProblem}

A palindrome is any string that is exactly the same as its reversal, like \texttt{I}, or \texttt{DEED}, or \texttt{RACECAR}, or \texttt{A\-MAN\-A\-PLAN\-A\-CAT\-A\-CANAL\-PANAMA}.

\begin{enumerate}[label=(\alph*)]
	\item Describe and analyze an algorithm to find the length of the \emph{longest subsequence} of a given string that is also a palindrome. For example, the longest palindrome subsequence of \texttt{\underline{M}A\underline{H}\-D\underline{Y}NA\underline{M}IC\-P\underline{RO}G\underline{R}AMZ\-LET\-\underline{M}E\-SHOW\-\underline{Y}OU\-T\underline{H}E\underline{M}} is \texttt{MHYMRORYHM}, so given that string as input, your algorithm should output the number 11. \setcounter{enumi}{2} 
	\item Any string can be decomposed into a sequence of palindromes. For example, the string \texttt{BUBBA\-SEES\-A\-BANANA} (``Bubba sees a banana.'') can be broken into palindromes in the following ways (and many others):
	
	\begin{center}
	\texttt{BUB} $\bullet$ \texttt{BASEESAB} $\bullet$ \texttt{ANANA}\\
	\texttt{B} $\bullet$ \texttt{U} $\bullet$ \texttt{BB} $\bullet$ \texttt{A} $\bullet$ \texttt{SEES} $\bullet$ \texttt{ABA} $\bullet$ \texttt{NAN} $\bullet$ \texttt{A}\\
	\texttt{B} $\bullet$ \texttt{U} $\bullet$ \texttt{BB} $\bullet$ \texttt{A} $\bullet$ \texttt{SEES} $\bullet$ \texttt{A} $\bullet$ \texttt{B} $\bullet$ \texttt{ANANA}\\
	\texttt{B} $\bullet$ \texttt{U} $\bullet$ \texttt{B} $\bullet$ \texttt{B} $\bullet$ \texttt{A} $\bullet$ \texttt{S} $\bullet$ \texttt{E} $\bullet$ \texttt{E} $\bullet$ \texttt{S} $\bullet$ \texttt{A} $\bullet$ \texttt{B} $\bullet$ \texttt{A} $\bullet$ \texttt{N} $\bullet$ \texttt{ANA}
	\end{center}
	Describe and analyze an efficient algorithm to find the smallest number of palindromes that make up a given input string. For example, given the input string \texttt{BUBBA\-SEES\-A\-BANANA}, your algorithm would return the integer 3.
\end{enumerate}

\solution\\

\part\\

\begin{algorithm}
	\caption{Longest Palindrome Subsequence}\label{algo:longpalinsubsq}
	\begin{algorithmic}[1]
		\Function{LPS}{$text[1..n]$}
			\State{$memorized \gets$ empty hash map}
			\State $memorized[$the empty list$] \gets 0$
			\For{$i \gets 1, n$}
				\For{$j \gets 1, n-i$}
					\State $subproblem \gets text[j..j+i]$
					\If{$i = 1$}
						\State $memorized[subproblem] \gets 1$ \Comment A single letter is always a palindrome of length one
					\Else
						\State $r \gets$ the maximum of $memorized[subproblem[2..i]]$ and $memorized[subproblem[1..i-1]$
						\If{$subproblem[1] = subproblem[i]$}
							\State $r \gets$ the maximum of $r$ and $2 + memorized[subproblem[2..i-1]$
						\EndIf
						\State $memorized[subproblem] \gets r$
					\EndIf
				\EndFor
			\EndFor
			\State \Return $memorized[text]$
		\EndFunction
	\end{algorithmic}
\end{algorithm}

This solution is similar to the previous algorithm. Each subproblem is a set of neighboring characters in the string, with $i$ ``layers'' of substring lengths and $1+n-i$ substrings of length $i$. Using the previous summation, we see that the runtime is $O(n^2)$.\\

\addtocounter{partCounter}{1}
\part\\



\end{homeworkProblem}

\pagebreak

\begin{homeworkProblem}
Suppose we are given a set $L$ of $n$ line segments in the plane, where each segment has one endpoint on the line $y = 0$ and one endpoint on the line $y = 1$, and all $2n$ endpoints are distinct. Describe and analyze an algorithm to compute the largest subset of $L$ in which no pair of segments intersects.\\

\solution\\



\end{homeworkProblem}

\pagebreak

\begin{homeworkProblem}

Oh, no! You have been appointed as the organizer of Giggle, Inc.'s annual mandatory holiday party! The employees at Giggle are organized into a strict hierarchy, that is, a tree with the company president at the root. The all-knowing oracles in Human Resources have assigned a real number to each employee measuring how ``fun'' the employee is. In order to keep things social, there is one restriction on the guest list: an employee cannot attend the party if their immediate supervisor is also present. On the other hand, the president of the company \emph{must} attend the party, even though she has a negative fun rating; it's her company, after all. Give an algorithm that makes a guest list for the party that maximizes the sum of the ``fun'' ratings of the guests.\\

\solution\\



\end{homeworkProblem}

\end{document}